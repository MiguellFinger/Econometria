% Options for packages loaded elsewhere
\PassOptionsToPackage{unicode}{hyperref}
\PassOptionsToPackage{hyphens}{url}
%
\documentclass[
]{article}
\usepackage{amsmath,amssymb}
\usepackage{lmodern}
\usepackage{ifxetex,ifluatex}
\ifnum 0\ifxetex 1\fi\ifluatex 1\fi=0 % if pdftex
  \usepackage[T1]{fontenc}
  \usepackage[utf8]{inputenc}
  \usepackage{textcomp} % provide euro and other symbols
\else % if luatex or xetex
  \usepackage{unicode-math}
  \defaultfontfeatures{Scale=MatchLowercase}
  \defaultfontfeatures[\rmfamily]{Ligatures=TeX,Scale=1}
\fi
% Use upquote if available, for straight quotes in verbatim environments
\IfFileExists{upquote.sty}{\usepackage{upquote}}{}
\IfFileExists{microtype.sty}{% use microtype if available
  \usepackage[]{microtype}
  \UseMicrotypeSet[protrusion]{basicmath} % disable protrusion for tt fonts
}{}
\makeatletter
\@ifundefined{KOMAClassName}{% if non-KOMA class
  \IfFileExists{parskip.sty}{%
    \usepackage{parskip}
  }{% else
    \setlength{\parindent}{0pt}
    \setlength{\parskip}{6pt plus 2pt minus 1pt}}
}{% if KOMA class
  \KOMAoptions{parskip=half}}
\makeatother
\usepackage{xcolor}
\IfFileExists{xurl.sty}{\usepackage{xurl}}{} % add URL line breaks if available
\IfFileExists{bookmark.sty}{\usepackage{bookmark}}{\usepackage{hyperref}}
\hypersetup{
  pdftitle={Trabalho Final},
  pdfauthor={Miguel Finger Schmidtke},
  hidelinks,
  pdfcreator={LaTeX via pandoc}}
\urlstyle{same} % disable monospaced font for URLs
\usepackage[margin=1in]{geometry}
\usepackage{color}
\usepackage{fancyvrb}
\newcommand{\VerbBar}{|}
\newcommand{\VERB}{\Verb[commandchars=\\\{\}]}
\DefineVerbatimEnvironment{Highlighting}{Verbatim}{commandchars=\\\{\}}
% Add ',fontsize=\small' for more characters per line
\usepackage{framed}
\definecolor{shadecolor}{RGB}{248,248,248}
\newenvironment{Shaded}{\begin{snugshade}}{\end{snugshade}}
\newcommand{\AlertTok}[1]{\textcolor[rgb]{0.94,0.16,0.16}{#1}}
\newcommand{\AnnotationTok}[1]{\textcolor[rgb]{0.56,0.35,0.01}{\textbf{\textit{#1}}}}
\newcommand{\AttributeTok}[1]{\textcolor[rgb]{0.77,0.63,0.00}{#1}}
\newcommand{\BaseNTok}[1]{\textcolor[rgb]{0.00,0.00,0.81}{#1}}
\newcommand{\BuiltInTok}[1]{#1}
\newcommand{\CharTok}[1]{\textcolor[rgb]{0.31,0.60,0.02}{#1}}
\newcommand{\CommentTok}[1]{\textcolor[rgb]{0.56,0.35,0.01}{\textit{#1}}}
\newcommand{\CommentVarTok}[1]{\textcolor[rgb]{0.56,0.35,0.01}{\textbf{\textit{#1}}}}
\newcommand{\ConstantTok}[1]{\textcolor[rgb]{0.00,0.00,0.00}{#1}}
\newcommand{\ControlFlowTok}[1]{\textcolor[rgb]{0.13,0.29,0.53}{\textbf{#1}}}
\newcommand{\DataTypeTok}[1]{\textcolor[rgb]{0.13,0.29,0.53}{#1}}
\newcommand{\DecValTok}[1]{\textcolor[rgb]{0.00,0.00,0.81}{#1}}
\newcommand{\DocumentationTok}[1]{\textcolor[rgb]{0.56,0.35,0.01}{\textbf{\textit{#1}}}}
\newcommand{\ErrorTok}[1]{\textcolor[rgb]{0.64,0.00,0.00}{\textbf{#1}}}
\newcommand{\ExtensionTok}[1]{#1}
\newcommand{\FloatTok}[1]{\textcolor[rgb]{0.00,0.00,0.81}{#1}}
\newcommand{\FunctionTok}[1]{\textcolor[rgb]{0.00,0.00,0.00}{#1}}
\newcommand{\ImportTok}[1]{#1}
\newcommand{\InformationTok}[1]{\textcolor[rgb]{0.56,0.35,0.01}{\textbf{\textit{#1}}}}
\newcommand{\KeywordTok}[1]{\textcolor[rgb]{0.13,0.29,0.53}{\textbf{#1}}}
\newcommand{\NormalTok}[1]{#1}
\newcommand{\OperatorTok}[1]{\textcolor[rgb]{0.81,0.36,0.00}{\textbf{#1}}}
\newcommand{\OtherTok}[1]{\textcolor[rgb]{0.56,0.35,0.01}{#1}}
\newcommand{\PreprocessorTok}[1]{\textcolor[rgb]{0.56,0.35,0.01}{\textit{#1}}}
\newcommand{\RegionMarkerTok}[1]{#1}
\newcommand{\SpecialCharTok}[1]{\textcolor[rgb]{0.00,0.00,0.00}{#1}}
\newcommand{\SpecialStringTok}[1]{\textcolor[rgb]{0.31,0.60,0.02}{#1}}
\newcommand{\StringTok}[1]{\textcolor[rgb]{0.31,0.60,0.02}{#1}}
\newcommand{\VariableTok}[1]{\textcolor[rgb]{0.00,0.00,0.00}{#1}}
\newcommand{\VerbatimStringTok}[1]{\textcolor[rgb]{0.31,0.60,0.02}{#1}}
\newcommand{\WarningTok}[1]{\textcolor[rgb]{0.56,0.35,0.01}{\textbf{\textit{#1}}}}
\usepackage{longtable,booktabs,array}
\usepackage{calc} % for calculating minipage widths
% Correct order of tables after \paragraph or \subparagraph
\usepackage{etoolbox}
\makeatletter
\patchcmd\longtable{\par}{\if@noskipsec\mbox{}\fi\par}{}{}
\makeatother
% Allow footnotes in longtable head/foot
\IfFileExists{footnotehyper.sty}{\usepackage{footnotehyper}}{\usepackage{footnote}}
\makesavenoteenv{longtable}
\usepackage{graphicx}
\makeatletter
\def\maxwidth{\ifdim\Gin@nat@width>\linewidth\linewidth\else\Gin@nat@width\fi}
\def\maxheight{\ifdim\Gin@nat@height>\textheight\textheight\else\Gin@nat@height\fi}
\makeatother
% Scale images if necessary, so that they will not overflow the page
% margins by default, and it is still possible to overwrite the defaults
% using explicit options in \includegraphics[width, height, ...]{}
\setkeys{Gin}{width=\maxwidth,height=\maxheight,keepaspectratio}
% Set default figure placement to htbp
\makeatletter
\def\fps@figure{htbp}
\makeatother
\setlength{\emergencystretch}{3em} % prevent overfull lines
\providecommand{\tightlist}{%
  \setlength{\itemsep}{0pt}\setlength{\parskip}{0pt}}
\setcounter{secnumdepth}{-\maxdimen} % remove section numbering
\ifluatex
  \usepackage{selnolig}  % disable illegal ligatures
\fi

\title{Trabalho Final}
\author{Miguel Finger Schmidtke}
\date{28/04/2022}

\begin{document}
\maketitle

\begin{longtable}[]{@{}
  >{\raggedright\arraybackslash}p{(\columnwidth - 0\tabcolsep) * \real{0.06}}@{}}
\toprule
\endhead
INTRODUÇÃO \\
Este trabalho se dedica a analisar a taxa de desemprego, causas
correlacionadas e qual o tamanho do impacto de cada uma delas, seja no
efeito de elevação da taxa de desemprego, ou o efeito de queda da taxa
do desemprego. Especificamente esta variável foi escolhida por ser uma
preocupação recorrente do governo brasileiro, está presente nos
discursos de todos os presidenciaveis, nas capas dos jornais e no dia a
dia da população. O Brasil teve um período relativamente tranquilo em
relação ao desemprego na década de 2001 a 2010. Na década seguinte,
passou a lidar com um aumento expressivo da taxa de desemprego. Durante
este período convivemos também com mudanças bruscas nas politicas
fiscais e monetárias. Cabe destacar aqui a Nova Matriz Econômica, que
teve seu inicio dado pela então presidente Dilma Rousseff, que tinha
como um de seus principais pilares a expansão da base monetária, altos
investimentos e pouca preocupação fiscal. O período que analisaremos
neste trabalho está compreendido entre os anos 2012 e 2021, totalizando
assim a análise de um período total de 10 anos. A escolha desse período
específico se deve ao entendimento de que a nova matriz economica e suas
principais implicações começariam a surtir efeito a partir do primeiro
ano em vigor. O entendimento sobre essa questão é que um período com
grandes mudanças nas variáveis de interesse poderia auxiliar no
desenvolvimento do trabalho, e evidenciar o impacto que mudanças bruscas
nestas políticas podem acarretar. Foi desenvolvida uma análise
econométrica com uma regressão linear múltipla, explicando a variação na
taxa de desemprego (Y) com as variáveis explanatórias Salário Mínimo
(X1), Taxa Selic (X2) e Variação IPCA(x3).O trabalho está dividido em
Introdução, uma breve revisão da literatura sobre os principais pontos
dos modelos de regressão linear múltipla, metodologia, resultados
discutidos e conclusão. É importante para a compreensão do trabalho o
entendimento sobre as variáveis de interesse. \\
\bottomrule
\end{longtable}

Revisão da Literatura\\
Modelos de regressão linear múltipla são os modelos que possuem duas ou
mais variáveis explicativas para a variável dependente. São lineares nos
parâmtreos. Precisam atender às seguintes condições\\
\textbf{(1) Modelo é linear nos parâmetros;}\\
\textbf{(2) Variáveis explicativas independem do termo de erro;}\\
\textbf{(3) A média do termo de erro é 0;}\\
\textbf{(4) Homocedasticidade;}\\
\textbf{(5) Ausência de Autocorrelação entre os termos;}\\
\textbf{(6) Número de observações maior que o de parâmetros;}\\
\textbf{(7) Deve haver variação nos valores das variáveis ``X'';}\\
\textbf{(8) Não há colinearidade exata entre as variáveis ``X'';}\\
\textbf{(9) Ausência de viés de especificação.}

Realizando a Regressão, teremos nos regressores estimados os efeitos da
variável explicativa correspondente no valor de Y. Estes regressores
estimados são obtidos com o método dos MínimosQuadrados Ordinários
(MQO), com o qual procuramos obter a menor Soma dos Quadrados dos
Resíduos(SQR) possível. Os resíduos da regressão podem ser uma medida
útil de quão bem a linha de regressão estimada se ajusta aos dados. Uma
boa equação de regressão é aquela que ajuda a explicar uma grande
proporção da variância de Y. Um modelo que tem resíduos muito grandes,
denuncia que a adequação é ruim, enquanto resíduo pequenos implicam uma
boa adequação.

Desta forma utilizamos a medida R\^{}2 que é a proporção da variação
total de Y(várivavel dependente) explicada pela regressão de Y contra X.
Esta medida R\^{}2 tem seu valor entre 0 e 1, um R\^{}2 igual a 0 ocorre
quando o modelo de regressão linear não ajuda em nada a explicar a
variação de Y.

METODOLOGIA

Os dados utilizados para essa abordagem foram retirados do WebSite
IPEADATA e das séries históricas disponibilizadas pelo IBGE.\\
As referências buscadas para este estudo com o auxilio da ferramenta de
dados do IPEA e do IBGE foram:

Taxa de Desemprego\\
O desemprego, de forma simplificada, se refere às pessoas com idade para
trabalhar (acima de 14 anos) que não estão trabalhando, mas estão
disponíveis e tentam encontrar trabalho. Assim, para alguém ser
considerado desempregado, não basta não possuir um emprego.(Fonte: IBGE)

Salário Mínimo

Salário mínimo nominal vigente - não considera abonos salariais
ocorridos nos períodos. O salário mínimo urbano foi instituído no Brasil
por decreto-lei do presidente Getúlio Vargas, durante a ditadura do
Estado Novo, e começou a vigorar em julho de 1940, com valores
diferenciados entre estados e sub-regiões. Em 1943, foi incorporado à
Consolidação das Leis do Trabalho (CLT) e, em 1963, foi estendido ao
campo por meio do Estatuto do Trabalhador Rural. Foi nacionalmente
unificado em maio de 1984, mas, desde 2000, a Lei Complementar 103
permite que os estados fixem pisos estaduais superiores ao mínimo
nacional. Fonte:IPEADATA/DIEESE

Taxa SELIC\\
A Selic é um dos elementos centrais da estratégia de política monetária
no Brasil, que está baseada em um sistema de metas de inflação. Criado
em 1999, ele estabelece o compromisso do país em adotar medidas para
manter a inflação dentro de uma faixa fixada periodicamente pelo
Conselho Monetário Nacional (CMN), composto pelos ministros e o
presidente do Banco Central. O objetivo é assegurar a estabilidade da
economia e evitar descontroles de preço como os que o país já viveu em
décadas passadas, que causam a perda do poder de compra da moeda.\\
Fonte: InfoMoney

Índice IPCA e Taxa de Variação\\
O Índice Nacional de Preços ao Consumidor Amplo (IPCA) mede a inflação
de um conjunto de bens e serviços comercializados no varejo, referentes
ao consumo pessoal das famílias, cujo rendimento varia entre 1 e 40
salários mínimos, visando uma cobertura de 90 \% das famílias
pertencentes as áreas urbanas de abrangência do Sistema Nacional de
Índices de Preços ao Consumidor (SNIPC), qualquer que seja a fonte de
rendimentos. É calculado a partir dos resultados dos índices regionais,
utilizando-se a média aritmética ponderada e cuja a variável de
ponderação é o Rendimento Familiar Monetário Disponível, tendo como
fonte de informação a Pesquisa de Orçamentos Familiares - POF. Refere-se
ao número índice (número de pontos ) das taxas de variações de preços do
grupo de bens e serviços.\\
Fonte: /IBGE

Período e análise\\
O período analisado (2012 - 2021) compreende o total de 10 anos
inteiros. A preparação das bases foi feita em CSV, todos os testes e a
sequência do trabalho foi realizado em RMarkDown.\\
A variável dependente que estamos investigando neste caso foi a Taxa de
Desemprego. Esta é uma variável de muita importância dado que a região
que exploramos é o Brasil, e este país convive, em sua história recente,
com uma taxa elevada de desemprego, e divide a opinião de economistas
sobre suas causas.\\
As variáveis explicativas foram as já mencionadas acima Salário Mínimo,
Taxa Selic e a variação do IPCA.\\
A análise aqui realizada tem o objetivo de testar as hipóteses
econômicas clássicas de que a taxa de desemprego pode ser influenciada
positivamente - o que neste caso significa mais desemprego - pelo
aumento do salário minimo, aumento da taxa de juros (SELIC), ou
diminuição da taxa de inflação - Neste caso consideramos a explicação de
Philipps, de que uma diminuição da inflação impacta a economia com o
aumento do desemprego.

TESTES E RESULTADOS

Iniciamos carregando a os pacotes, e na sequência a base de dados

\begin{Shaded}
\begin{Highlighting}[]
\FunctionTok{library}\NormalTok{(car)                                                                                   }
\end{Highlighting}
\end{Shaded}

\begin{verbatim}
## Loading required package: carData
\end{verbatim}

\begin{Shaded}
\begin{Highlighting}[]
\FunctionTok{library}\NormalTok{(carData)                                                                         }
\FunctionTok{library}\NormalTok{(dplyr)                                                                           }
\end{Highlighting}
\end{Shaded}

\begin{verbatim}
## 
## Attaching package: 'dplyr'
\end{verbatim}

\begin{verbatim}
## The following object is masked from 'package:car':
## 
##     recode
\end{verbatim}

\begin{verbatim}
## The following objects are masked from 'package:stats':
## 
##     filter, lag
\end{verbatim}

\begin{verbatim}
## The following objects are masked from 'package:base':
## 
##     intersect, setdiff, setequal, union
\end{verbatim}

\begin{Shaded}
\begin{Highlighting}[]
\FunctionTok{library}\NormalTok{(knitr)                                                                           }
\FunctionTok{library}\NormalTok{(lmtest)                                                                          }
\end{Highlighting}
\end{Shaded}

\begin{verbatim}
## Loading required package: zoo
\end{verbatim}

\begin{verbatim}
## 
## Attaching package: 'zoo'
\end{verbatim}

\begin{verbatim}
## The following objects are masked from 'package:base':
## 
##     as.Date, as.Date.numeric
\end{verbatim}

\begin{Shaded}
\begin{Highlighting}[]
\FunctionTok{library}\NormalTok{(pacman)                                                                          }
\FunctionTok{library}\NormalTok{(psych)                                                                           }
\end{Highlighting}
\end{Shaded}

\begin{verbatim}
## 
## Attaching package: 'psych'
\end{verbatim}

\begin{verbatim}
## The following object is masked from 'package:car':
## 
##     logit
\end{verbatim}

\begin{Shaded}
\begin{Highlighting}[]
\NormalTok{Base }\OtherTok{\textless{}{-}} \FunctionTok{read.csv2}\NormalTok{(}\StringTok{\textquotesingle{}Pasta2.csv\textquotesingle{}}\NormalTok{)}
\end{Highlighting}
\end{Shaded}

\begin{verbatim}
                                                                                                                                                                            Validamos a Base de dados                                                                           
\end{verbatim}

\begin{Shaded}
\begin{Highlighting}[]
\FunctionTok{View}\NormalTok{ (Base)                                                                                                                                                             }
\end{Highlighting}
\end{Shaded}

Utilizamos a função glimpse que permite visualizar os dados com os
atributos como linhas e os objetos como colunas

\begin{Shaded}
\begin{Highlighting}[]
\FunctionTok{glimpse}\NormalTok{ (Base)                                                       }
\end{Highlighting}
\end{Shaded}

\begin{verbatim}
## Rows: 120
## Columns: 5
## $ Data     <chr> "jan/12", "fev/12", "mar/12", "abr/12", "mai/12", "jun/12", "~
## $ TX_DES   <dbl> 8.0, 7.8, 7.7, 7.6, 7.5, 7.4, 7.1, 7.0, 6.8, 6.9, 7.3, 7.8, 8~
## $ SL_MIN   <int> 622, 622, 622, 622, 622, 622, 622, 622, 622, 622, 622, 622, 6~
## $ TX_SLC   <dbl> 0.89, 0.75, 0.82, 0.71, 0.74, 0.64, 0.68, 0.69, 0.54, 0.61, 0~
## $ VAR_IPCA <dbl> 0.56, 0.45, 0.21, 0.64, 0.36, 0.08, 0.43, 0.41, 0.57, 0.59, 0~
\end{verbatim}

As colunas ficaram definidas como:

TX\_DES SL\_MIN SLC VAR\_IPCA

Sendo, respectivamente: Taxa de Desemprego, Salário Mínimo, Taxa Selic,
Variação do IPCA.

Utilizando a função lm criei o modelo de regressão linear multipla
chamado aqui de regl.

\begin{Shaded}
\begin{Highlighting}[]
\NormalTok{regl }\OtherTok{\textless{}{-}} \FunctionTok{lm}\NormalTok{(TX\_DES }\SpecialCharTok{\textasciitilde{}}\NormalTok{ SL\_MIN }\SpecialCharTok{+}\NormalTok{ TX\_SLC }\SpecialCharTok{+}\NormalTok{ VAR\_IPCA ,   }\AttributeTok{data =}\NormalTok{ Base)                                                                                                                                  }
\end{Highlighting}
\end{Shaded}

Para verificar a relevância estatistica do modelo, cria-se um resumo das
estatisticas a partir do modelo já estabelecido.

\begin{Shaded}
\begin{Highlighting}[]
\NormalTok{sregl }\OtherTok{\textless{}{-}} \FunctionTok{summary}\NormalTok{(regl)}
\NormalTok{sregl                                                                                                                             }
\end{Highlighting}
\end{Shaded}

\begin{verbatim}
## 
## Call:
## lm(formula = TX_DES ~ SL_MIN + TX_SLC + VAR_IPCA, data = Base)
## 
## Residuals:
##      Min       1Q   Median       3Q      Max 
## -2.39124 -0.84262  0.07158  0.78037  2.32716 
## 
## Coefficients:
##               Estimate Std. Error t value Pr(>|t|)    
## (Intercept) -4.2729649  0.8689280  -4.918 2.91e-06 ***
## SL_MIN       0.0167797  0.0007729  21.709  < 2e-16 ***
## TX_SLC       1.0381300  0.3879419   2.676  0.00853 ** 
## VAR_IPCA    -0.8556017  0.2739171  -3.124  0.00226 ** 
## ---
## Signif. codes:  0 '***' 0.001 '**' 0.01 '*' 0.05 '.' 0.1 ' ' 1
## 
## Residual standard error: 1.071 on 116 degrees of freedom
## Multiple R-squared:  0.8445, Adjusted R-squared:  0.8405 
## F-statistic: 209.9 on 3 and 116 DF,  p-value: < 2.2e-16
\end{verbatim}

Vamos iniciar observando o teste F, tendo valor P \textless{} 0,05
indica que nosso modelo de regressão é estatisticamente significativo
para explicar a variação e fazer previsões.

Podemos observar também que o nosso R\^{}2 nos mostra que as variáveis
explicativas explicam em até 84,45\% a variação da variável dependente.

O maior problema na interpretação do R\^{}2 como medida de ajustamento
da regressão, se dá pelo fato de que cada variável que implementamos a
mais no modelo, necessariamente aumentará seu valor. Desta forma se
mostra muito importante a interpretação do R\^{}2 ajustado.

O R\^{}2 ajustado leva em consideração o aumento da quantidade de
estimadores, e pode aumentar o diminuir com estes. Nosso R\^{}2 prova a
consistência do modelo pois se mantém com valor acima de 50\%, que é um
valor de refêrencia para estudos econômicos.

Tendo validado a significância do modelo em geral, podemos começar a
olhar as variáveis separadamente, para verificar o que cada uma delas
pode nos dizer sobre a variação da nossa variável dependente, que neste
caso é a taxa de desemprego.

Vamos iniciar olhando o teste t, vale lembrar que aqui temos como
hipótese nula que o coeficiente é =0 ou seja, não estatisticamente
significativo, e hipótese alternativa coeficiente diferente de 0, ou
seja estatisticamente signitifativo. Desta forma todo valor
P\textgreater0,05 iguala, estatisticamente, a 0. Todo valor
P\textless0,05 mostra uma significância estatistica da variável na
explicação da variável dependente.

Das três variáveis de escolha podemos notar que a que se provou mais
estatisticamente significativa na explicação da variação da taxa de
desemprego foi o salário mínimo.

A váriavel explicativa ``TX\_SLC'' se provou estatisticamente
significativa na explicação da variação da taxa de desemprego.

A variável que mostra a variação do IPCA também é estatisticamente
significativa para explicar a variação da taxa de desemprego.

Partindo agora para a explicação dos resultados das variáveis
explicativas, vamos começar analisando o Salário mínimo e seu impacto na
taxa de desemprego. O modelo estimou que para cada Real(R\$) de aumento
no salário mínimo, nós temos aproximadamente 0,017 de aumento na taxa de
desemprego.

Este é um resultado muito interessante para nós, e joga luz sobre uma
discussão presente atualmente no Brasil, que é a de que a manutenção do
salário mínimo contribui para a manutenção do desemprego, neste caso, de
forma positiva, mostrando que um aumento no salário minimo tem
correlação com um aumento na taxa de desemprego.

Analisando o resultado da variável TX\_SLC, que é a taxa Selic,
encontramos uma relação positiva, que neste caso corrobora a tese
difundida informalmente entre as pessoas de que um aumento na taxa de
juros, diminui o consumo das familias e desta forma aumentaria o
desemprego. Podemos afirmar a correlação entre estas variáveis.

Analisando a última variável explicativa, que é a Variação do IPCA,
encontramos uma relação negativa, que é exatamente o ponto levantado por
Philips e depois complementado por hayek. A inflação afetar a taxa de
desemprego de forma negativa significa que cada aumento da inflação
diminui a taxa de desemprego, e consequentemente, cada diminuição da
taxa de inflação acarreta um aumento da taxa de desemprego. Aqui
encontramos essa implicação com a seguinte proporção: para cada
diminuição de 1\% no índice de inflação, temos o aumento de 0,85\% na
taxa de desemprego.

Agora que já entendemos o modelo econométrico, vou realizar as análises
estatisticas tanto do modelo quanto da base de dados, para conferir se
não há nenhum problema com o nosso estudo.

Fazemos o plot com a função:

\begin{Shaded}
\begin{Highlighting}[]
\FunctionTok{plot}\NormalTok{(regl)}
\end{Highlighting}
\end{Shaded}

\includegraphics{Trabalho-Final-Econometria_files/figure-latex/unnamed-chunk-6-1.pdf}
\includegraphics{Trabalho-Final-Econometria_files/figure-latex/unnamed-chunk-6-2.pdf}
\includegraphics{Trabalho-Final-Econometria_files/figure-latex/unnamed-chunk-6-3.pdf}
\includegraphics{Trabalho-Final-Econometria_files/figure-latex/unnamed-chunk-6-4.pdf}
Iniciamos Analisando o primeiro gráfico de nome ``Residuals vs Fitted''
Este gráfico nos mostra os resíduos x valores ajustados, e nos ajuda a
entender a linearidade da nossa regressão. Quanto mais a linha vermelha
estiver próxima da linha pontilhada mais linear é nosso modelo. Neste
caso a linha está satisfatoriamente horizontal.

No segundo gráfico de nome ``Normal Q-Q'' Tem no eixo Y os resíduos
padronizados e no eixo X os residuos teóricos para uma distribuição
normal. A qualidade do ajustamento dos resíduos a linha pontilhada nos
mostra a normalidade da distribuição.

Terceiro gráfico de nome ``Scale-Location'' Teste de homocedasticidade,
neste teste precisamos observar um padrão retangular da distribuição dos
resíduos, o que é confirmado pelo gráfico gerado a partir do modelo.

Quarto gráfico de nome ``Residual vs Leverage'' Nos ajuda com a
verificação de Outliers. É esperado que a escala Y esteja dentro do
padrão -3 \textless{} y \textless{} +3, o que podemos confirmar neste
gráfico. Caso existissem outliers veríamos uma linha vermelha, abaixo ou
acima dos demais residuos, e o outlier posicionado após esta linha.

A análise gráfica é extremamente produtiva e nos ajuda com a verificação
dos pressupostos do modelo, mas tem a fragilidade de deixar a análise
muito subjetiva de acordo com o pesquisador.

Deste modo começarei a introduzir alguns testes que podem nos ajudar a
confirmar os pressupostos verificados nos graficos anteriores.

Para verificar a normalidade vamos utilizar a função

Que caso retorno um valor p\textgreater0,05 confirma a distribuição
normal

\begin{Shaded}
\begin{Highlighting}[]
\FunctionTok{shapiro.test}\NormalTok{(regl}\SpecialCharTok{$}\NormalTok{residuals)}
\end{Highlighting}
\end{Shaded}

\begin{verbatim}
## 
##  Shapiro-Wilk normality test
## 
## data:  regl$residuals
## W = 0.98297, p-value = 0.1343
\end{verbatim}

Como podemos verificar 0,1343 \textgreater{} 0,05 confirmando a
normalidade da distribuição

Para verificar a presença de outliers utilizaremos a função

Esperamos que os valores mínimo e máximo estejam dentro do padrão -3
\textless{} x \textless{} +3.

\begin{Shaded}
\begin{Highlighting}[]
\FunctionTok{summary}\NormalTok{(}\FunctionTok{rstandard}\NormalTok{(regl))                                                                          }
\end{Highlighting}
\end{Shaded}

\begin{verbatim}
##      Min.   1st Qu.    Median      Mean   3rd Qu.      Max. 
## -2.288463 -0.801135  0.067675  0.001322  0.743091  2.264854
\end{verbatim}

Como podemos verificar, o teste confirma que não há outliers que
precisem ser retirados do modelo.

Para verificação da multicolinearidade da nossa base de dados vamos
utilizar um painel, com a função:

\begin{Shaded}
\begin{Highlighting}[]
\FunctionTok{pairs.panels}\NormalTok{(Base)}
\end{Highlighting}
\end{Shaded}

\includegraphics{Trabalho-Final-Econometria_files/figure-latex/unnamed-chunk-9-1.pdf}

O único número que aparece neste painel que está acima de 0,8, o que
poderia indicar um problema, está relacionando a variável dependente à
variável explicativa, e nós só temos interesse em analisar a
multicolineariedade entre os valores estimados.

Desta forma entendemos atráves deste painel que não há problema de
multicolinearidade.

Podemos utilizar outro método de verificação para que possamos afirmar
com mais segurança que não temos um problema de multicolinearidade.

Utilizamos aqui a função

\begin{Shaded}
\begin{Highlighting}[]
\FunctionTok{vif}\NormalTok{(regl)}
\end{Highlighting}
\end{Shaded}

\begin{verbatim}
##   SL_MIN   TX_SLC VAR_IPCA 
## 1.485176 1.481894 1.002874
\end{verbatim}

Esta função permite identificar o problema diretamente no modelo,
diferentemente do painel apresentado acima.

Para implicar em um problema de multicolinearidade o valor da variavel
deve estar acima de 10, e como podemos verificar no teste acima, nenhuma
de nossas variáveis apresenta este valor.

Decidi realizar também o intervalo de confiança para assegurar a
significância estatistica das variaveis

utilizamos então a função

\begin{Shaded}
\begin{Highlighting}[]
\FunctionTok{confint}\NormalTok{(regl)}
\end{Highlighting}
\end{Shaded}

\begin{verbatim}
##                   2.5 %      97.5 %
## (Intercept) -5.99398621 -2.55194361
## SL_MIN       0.01524881  0.01831056
## TX_SLC       0.26976225  1.80649775
## VAR_IPCA    -1.39812901 -0.31307442
\end{verbatim}

Este resultado provará a falta de significância estatistica caso o
intervalo apresentado passe por 0.\\
Podemos observar que todos os intervalos estão inteiramente acima ou
abaixo de 0, nenhum deles compreende exatamente o número 0, e desta
forma temos segurança para reafirmar a significância estatistica do
modelo.

Para verifiar presença de correlação serial

vamos utilizar o teste Durbin-Watson

\begin{Shaded}
\begin{Highlighting}[]
\FunctionTok{dwtest}\NormalTok{(regl)}
\end{Highlighting}
\end{Shaded}

\begin{verbatim}
## 
##  Durbin-Watson test
## 
## data:  regl
## DW = 0.17183, p-value < 2.2e-16
## alternative hypothesis: true autocorrelation is greater than 0
\end{verbatim}

A baixa estatistica Durbin-Watson(DW=0,1718) é um forte indicio da
presença de correlação serial de primeira ordem.

Por fim, vamos comparar os dados reais com o resultado do modelo, ou
seja, comparar os dados históricos da taxa de desemprego com os dados de
taxa de desemprego entregues pelo modelo quando inseridos Salário
Minimo, Taxa Selic, e Variação de IPCA.

\begin{Shaded}
\begin{Highlighting}[]
\NormalTok{EST\_REG }\OtherTok{\textless{}{-}} \FunctionTok{fitted.values}\NormalTok{(regl)                                                       }
\NormalTok{EST\_REG}
\end{Highlighting}
\end{Shaded}

\begin{verbatim}
##         1         2         3         4         5         6         7         8 
##  6.608798  6.557576  6.835590  6.353486  6.624199  6.759954  6.502019  6.529512 
##         9        10        11        12        13        14        15        16 
##  6.236897  6.292454  6.221610  6.059045  6.990722  7.098984  7.272500  7.266340 
##        17        18        19        20        21        22        23        24 
##  7.409967  7.514465  7.825447  7.635390  7.541273  7.456854  7.389090  7.136631 
##        25        26        27        28        29        30        31        32 
##  8.287357  8.105285  7.887734  8.153541  8.385123  8.384553  8.853195  8.564800 
##        33        34        35        36        37        38        39        40 
##  8.332532  8.502398  8.311199  8.204763  8.864323  8.756860  8.899688  9.328173 
##        41        42        43        44        45        46        47        48 
##  9.344030  9.384301  9.643947  9.913519  9.639726  9.400158  9.185687  9.332280 
##        49        50        51        52        53        54        55        56 
## 10.506962 10.761247 11.329480 11.071659 10.978113 11.397928 11.200569 11.383212 
##        57        58        59        60        61        62        63        64 
## 11.577034 11.360738 11.418805 11.399183 12.256033 12.070425 12.325736 12.149939 
##        65        66        67        68        69        70        71        72 
## 12.149824 12.487274 12.074760 12.117540 11.977107 11.754651 11.801766 11.633726 
##        73        74        75        76        77        78        79        80 
## 12.088846 11.948983 12.208060 12.086450 11.932442 11.196624 12.013096 12.403593 
##        81        82        83        84        85        86        87        88 
## 11.812087 11.910424 12.423215 12.115198 12.759959 12.613936 12.319381 12.525296 
##        89        90        91        92        93        94        95        96 
## 12.922523 12.952526 12.902331 12.898110 12.984925 12.885903 12.431293 11.873327 
##        97        98        99       100       101       102       103       104 
## 13.375941 13.348963 13.554878 13.817719 13.836086 13.257357 13.151034 13.222562 
##       105       106       107       108       109       110       111       112 
## 12.880322 12.692089 12.656040 12.272845 14.126508 13.583828 13.596605 14.137460 
##       113       114       115       116       117       118       119       120 
## 13.754834 14.053040 13.737038 13.824423 13.316766 13.333194 13.403011 13.591243
\end{verbatim}

Com os valores originais

\begin{Shaded}
\begin{Highlighting}[]
\FunctionTok{print}\NormalTok{(Base}\SpecialCharTok{$}\NormalTok{TX\_DES)}
\end{Highlighting}
\end{Shaded}

\begin{verbatim}
##   [1]  8.0  7.8  7.7  7.6  7.5  7.4  7.1  7.0  6.8  6.9  7.3  7.8  8.1  7.9  7.7
##  [16]  7.5  7.4  7.2  7.0  6.8  6.6  6.3  6.5  6.8  7.2  7.2  7.1  6.9  7.0  7.0
##  [31]  6.9  6.7  6.6  6.6  6.9  7.5  8.0  8.1  8.3  8.4  8.7  8.9  9.0  9.1  9.1
##  [46]  9.1  9.6 10.3 11.1 11.3 11.3 11.4 11.7 11.9 11.9 11.9 12.0 12.2 12.7 13.3
##  [61] 13.9 13.7 13.4 13.1 12.9 12.7 12.5 12.3 12.1 11.9 12.3 12.7 13.2 13.0 12.8
##  [76] 12.6 12.4 12.3 12.0 11.9 11.7 11.7 12.2 12.6 12.8 12.6 12.4 12.1 12.0 11.9
##  [91] 11.9 11.8 11.3 11.1 11.4 11.8 12.4 12.7 13.1 13.6 14.1 14.8 14.9 14.6 14.4
## [106] 14.2 14.5 14.6 14.9 14.8 14.7 14.2 13.7 13.1 12.6 12.1 11.6 11.1 11.2 11.2
\end{verbatim}

Para poder colocar estes valores em gráficos e analisar seu
comportamento será necessário um ajuste na base de dados:

Inicio carregando um arquivo com as mesmas informações, apenas uma
coluna nova para data:

\begin{Shaded}
\begin{Highlighting}[]
\NormalTok{dados1 }\OtherTok{\textless{}{-}} \FunctionTok{read.csv2}\NormalTok{(}\AttributeTok{file =} \StringTok{"dados1.csv"}\NormalTok{, }\AttributeTok{header =} \ConstantTok{TRUE}\NormalTok{)}
\FunctionTok{head}\NormalTok{(dados1)}
\end{Highlighting}
\end{Shaded}

\begin{verbatim}
##    data2   Data TX_DES SL_MIN TX_SLC VAR_IPCA
## 1 Jan-12 Jan-12    8.0    622   0.89     0.56
## 2 Feb-12 fev/12    7.8    622   0.75     0.45
## 3 Mar-12 Mar-12    7.7    622   0.82     0.21
## 4 Apr-12 abr/12    7.6    622   0.71     0.64
## 5 May-12 mai/12    7.5    622   0.74     0.36
## 6 Jun-12 Jun-12    7.4    622   0.64     0.08
\end{verbatim}

E também a utilização dos pacotes necessários:

\begin{Shaded}
\begin{Highlighting}[]
\FunctionTok{library}\NormalTok{(dplyr)}
\FunctionTok{library}\NormalTok{(lubridate)}
\end{Highlighting}
\end{Shaded}

\begin{verbatim}
## 
## Attaching package: 'lubridate'
\end{verbatim}

\begin{verbatim}
## The following objects are masked from 'package:base':
## 
##     date, intersect, setdiff, union
\end{verbatim}

Vamos utilizar funções para criar uma sequência de datas que serão
incorporadas à base de dados:

\begin{Shaded}
\begin{Highlighting}[]
\NormalTok{data\_inicial }\OtherTok{\textless{}{-}} \FunctionTok{ymd}\NormalTok{(}\StringTok{"2012{-}01{-}01"}\NormalTok{)}
\NormalTok{data\_final }\OtherTok{\textless{}{-}} \FunctionTok{ymd}\NormalTok{(}\StringTok{"2021{-}12{-}01"}\NormalTok{)}
\NormalTok{dados1}\SpecialCharTok{$}\NormalTok{data4 }\OtherTok{\textless{}{-}} \FunctionTok{seq}\NormalTok{(data\_inicial, data\_final, }\StringTok{"month"}\NormalTok{)}
\FunctionTok{head}\NormalTok{(dados1)}
\end{Highlighting}
\end{Shaded}

\begin{verbatim}
##    data2   Data TX_DES SL_MIN TX_SLC VAR_IPCA      data4
## 1 Jan-12 Jan-12    8.0    622   0.89     0.56 2012-01-01
## 2 Feb-12 fev/12    7.8    622   0.75     0.45 2012-02-01
## 3 Mar-12 Mar-12    7.7    622   0.82     0.21 2012-03-01
## 4 Apr-12 abr/12    7.6    622   0.71     0.64 2012-04-01
## 5 May-12 mai/12    7.5    622   0.74     0.36 2012-05-01
## 6 Jun-12 Jun-12    7.4    622   0.64     0.08 2012-06-01
\end{verbatim}

Vamos carregar o pacote especifico para a criação dos gráficos
comparativos:

\begin{Shaded}
\begin{Highlighting}[]
\FunctionTok{library}\NormalTok{(ggplot2)}
\end{Highlighting}
\end{Shaded}

\begin{verbatim}
## 
## Attaching package: 'ggplot2'
\end{verbatim}

\begin{verbatim}
## The following objects are masked from 'package:psych':
## 
##     %+%, alpha
\end{verbatim}

E por fim, fazemos o Plot do gráfico da taxa de desemprego retirada da
série histórica:

\begin{Shaded}
\begin{Highlighting}[]
\NormalTok{g1 }\OtherTok{\textless{}{-}} \FunctionTok{ggplot}\NormalTok{(dados1, }\FunctionTok{aes}\NormalTok{(data4, TX\_DES)) }\SpecialCharTok{+} 
  \FunctionTok{geom\_line}\NormalTok{() }\SpecialCharTok{+}
  \FunctionTok{xlab}\NormalTok{(}\StringTok{"meses"}\NormalTok{) }\SpecialCharTok{+} 
  \FunctionTok{ylab}\NormalTok{(}\StringTok{"Variação da Taxa de Desemprego"}\NormalTok{) }\SpecialCharTok{+} 
  \FunctionTok{scale\_x\_date}\NormalTok{(}\AttributeTok{date\_breaks =} \StringTok{"12 month"}\NormalTok{, }\AttributeTok{date\_labels =} \StringTok{"\%b}\SpecialCharTok{\textbackslash{}n}\StringTok{\%Y"}\NormalTok{)}

\NormalTok{g1}
\end{Highlighting}
\end{Shaded}

\includegraphics{Trabalho-Final-Econometria_files/figure-latex/unnamed-chunk-19-1.pdf}

Agora para efeito de comparação, vamos gerar o mesmo gráfico para os
valores estimados:

\begin{Shaded}
\begin{Highlighting}[]
\NormalTok{g2 }\OtherTok{\textless{}{-}} \FunctionTok{ggplot}\NormalTok{(dados1, }\FunctionTok{aes}\NormalTok{(data4, EST\_REG)) }\SpecialCharTok{+} 
  \FunctionTok{geom\_line}\NormalTok{() }\SpecialCharTok{+}
  \FunctionTok{xlab}\NormalTok{(}\StringTok{"meses"}\NormalTok{) }\SpecialCharTok{+} 
  \FunctionTok{ylab}\NormalTok{(}\StringTok{"Variação da Taxa de Desemprego"}\NormalTok{) }\SpecialCharTok{+} 
  \FunctionTok{scale\_x\_date}\NormalTok{(}\AttributeTok{date\_breaks =} \StringTok{"12 month"}\NormalTok{, }\AttributeTok{date\_labels =} \StringTok{"\%b}\SpecialCharTok{\textbackslash{}n}\StringTok{\%Y"}\NormalTok{)}

\NormalTok{g2}
\end{Highlighting}
\end{Shaded}

\includegraphics{Trabalho-Final-Econometria_files/figure-latex/unnamed-chunk-20-1.pdf}

Podemos notar que os suportes em 6, 12,5 e 15 são respeitados por ambas
as séries. O modelo compreende bem as direções apontadas e confirma a
importância das três variáveis destacadas para o entendimento da taxa de
desemprego.

CONCLUSÃO

Com este trabalho pudemos confirmar as relações que já estão difundidas
no campo acadêmico.\\
É importante destacar que a correlação pressuposta pelo modelo não
implica em causalidade. Podemos investigar alguns pontos importantes
para entender esta correlação. Foi possível confirmar a correlação
positiva entre aumento de salário mínimo e o aumento da taxa de
desemprego. A possível causalidade entre aumento do salário mínimo e
aumento na taxa de desemprego, se deve à produtividade dos individuos
que não é a mesma, e podem estar abaixo do nível minimo de salário,
desta forma estes individuos incorreriam em custos excessivos para as
empresas e acabam sem emprego.

Assim como também percebemos a relação positiva entre aumento da taxa
selic e aumento da taxa de desemprego, a taxa selic neste estudo
impactou mais que a variação do salário mínimo. A possível causalidade
aqui seria explicada por uma queda no consumo e aumento da disposição a
poupança dos individuos, o que faz as empresas revisarem suas
estrategias de investimento, uma vez que podem observar queda em suas
receitas.\\
Seria possível também observar que o crédito para as empresas ficaria
mais dispendioso, obrigando-as a cortar custos.

Podemos verificar também a correlação negativa entre taxa de inflação e
taxa de desemprego, demonstrando que um aumento na taxa de inflação pode
incorrer em uma queda na taxa de desemprego.\\
Importante destacar que o modelo neste caso pressupõe que o aumento na
taxa de inflação geraria uma queda na taxa de desemprego, porém, não
podemos afirmar que esses empregos se manteriam ao longo do tempo apenas
pela alta taxa de inflação.\\
O que poderíamos afirmar neste caso, seria que enquanto estiver
aumentando a taxa de inflação, vamos observar quedas na taxa de
desemprego.

Referências\\
GUJARATI, N. D. Econometria básica. 5.ed. New York, EUA: Amgh, 2011.\\
PINDYCK, Robert S.; RUBINFELD, Daniel L. ECONOMETRIA Modelos e
Previsões. 4.ed.Rio de Janeiro, BR: Elsevier Editora, 2004.\\
IBGE: Instituto Brasileiro de Geografia e Estatistica, Disponível em:
\url{https://www.ibge.gov.br/estatisticas/todos-os-produtos-estatisticas.html}.
Acesso em: 29/04/2022. IPEADATA. Disponível em:
\url{http://www.ipeadata.gov.br/Default.aspx}. Acesso em: 28/04/2022.
Infomoney, Disponível em: \url{https://www.infomoney.com.br/}. Acesso
em: 29/04/2022.

\end{document}
